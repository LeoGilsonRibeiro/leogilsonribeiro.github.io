% Options for packages loaded elsewhere
\PassOptionsToPackage{unicode}{hyperref}
\PassOptionsToPackage{hyphens}{url}
\PassOptionsToPackage{dvipsnames,svgnames,x11names}{xcolor}
%
\documentclass[
  letterpaper,
  DIV=11,
  numbers=noendperiod]{scrartcl}

\usepackage{amsmath,amssymb}
\usepackage{iftex}
\ifPDFTeX
  \usepackage[T1]{fontenc}
  \usepackage[utf8]{inputenc}
  \usepackage{textcomp} % provide euro and other symbols
\else % if luatex or xetex
  \usepackage{unicode-math}
  \defaultfontfeatures{Scale=MatchLowercase}
  \defaultfontfeatures[\rmfamily]{Ligatures=TeX,Scale=1}
\fi
\usepackage{lmodern}
\ifPDFTeX\else  
    % xetex/luatex font selection
\fi
% Use upquote if available, for straight quotes in verbatim environments
\IfFileExists{upquote.sty}{\usepackage{upquote}}{}
\IfFileExists{microtype.sty}{% use microtype if available
  \usepackage[]{microtype}
  \UseMicrotypeSet[protrusion]{basicmath} % disable protrusion for tt fonts
}{}
\makeatletter
\@ifundefined{KOMAClassName}{% if non-KOMA class
  \IfFileExists{parskip.sty}{%
    \usepackage{parskip}
  }{% else
    \setlength{\parindent}{0pt}
    \setlength{\parskip}{6pt plus 2pt minus 1pt}}
}{% if KOMA class
  \KOMAoptions{parskip=half}}
\makeatother
\usepackage{xcolor}
\setlength{\emergencystretch}{3em} % prevent overfull lines
\setcounter{secnumdepth}{-\maxdimen} % remove section numbering
% Make \paragraph and \subparagraph free-standing
\ifx\paragraph\undefined\else
  \let\oldparagraph\paragraph
  \renewcommand{\paragraph}[1]{\oldparagraph{#1}\mbox{}}
\fi
\ifx\subparagraph\undefined\else
  \let\oldsubparagraph\subparagraph
  \renewcommand{\subparagraph}[1]{\oldsubparagraph{#1}\mbox{}}
\fi


\providecommand{\tightlist}{%
  \setlength{\itemsep}{0pt}\setlength{\parskip}{0pt}}\usepackage{longtable,booktabs,array}
\usepackage{calc} % for calculating minipage widths
% Correct order of tables after \paragraph or \subparagraph
\usepackage{etoolbox}
\makeatletter
\patchcmd\longtable{\par}{\if@noskipsec\mbox{}\fi\par}{}{}
\makeatother
% Allow footnotes in longtable head/foot
\IfFileExists{footnotehyper.sty}{\usepackage{footnotehyper}}{\usepackage{footnote}}
\makesavenoteenv{longtable}
\usepackage{graphicx}
\makeatletter
\def\maxwidth{\ifdim\Gin@nat@width>\linewidth\linewidth\else\Gin@nat@width\fi}
\def\maxheight{\ifdim\Gin@nat@height>\textheight\textheight\else\Gin@nat@height\fi}
\makeatother
% Scale images if necessary, so that they will not overflow the page
% margins by default, and it is still possible to overwrite the defaults
% using explicit options in \includegraphics[width, height, ...]{}
\setkeys{Gin}{width=\maxwidth,height=\maxheight,keepaspectratio}
% Set default figure placement to htbp
\makeatletter
\def\fps@figure{htbp}
\makeatother

\KOMAoption{captions}{tableheading}
\makeatletter
\@ifpackageloaded{caption}{}{\usepackage{caption}}
\AtBeginDocument{%
\ifdefined\contentsname
  \renewcommand*\contentsname{Table of contents}
\else
  \newcommand\contentsname{Table of contents}
\fi
\ifdefined\listfigurename
  \renewcommand*\listfigurename{List of Figures}
\else
  \newcommand\listfigurename{List of Figures}
\fi
\ifdefined\listtablename
  \renewcommand*\listtablename{List of Tables}
\else
  \newcommand\listtablename{List of Tables}
\fi
\ifdefined\figurename
  \renewcommand*\figurename{Figure}
\else
  \newcommand\figurename{Figure}
\fi
\ifdefined\tablename
  \renewcommand*\tablename{Table}
\else
  \newcommand\tablename{Table}
\fi
}
\@ifpackageloaded{float}{}{\usepackage{float}}
\floatstyle{ruled}
\@ifundefined{c@chapter}{\newfloat{codelisting}{h}{lop}}{\newfloat{codelisting}{h}{lop}[chapter]}
\floatname{codelisting}{Listing}
\newcommand*\listoflistings{\listof{codelisting}{List of Listings}}
\makeatother
\makeatletter
\makeatother
\makeatletter
\@ifpackageloaded{caption}{}{\usepackage{caption}}
\@ifpackageloaded{subcaption}{}{\usepackage{subcaption}}
\makeatother
\ifLuaTeX
  \usepackage{selnolig}  % disable illegal ligatures
\fi
\usepackage{bookmark}

\IfFileExists{xurl.sty}{\usepackage{xurl}}{} % add URL line breaks if available
\urlstyle{same} % disable monospaced font for URLs
\hypersetup{
  pdftitle={O mendigo, o moleque e o malandro de João Antônio estão de volta. Saindo do submundo para a galeria dos heróis marginais},
  pdfauthor={Leo Gilson Ribeiro},
  colorlinks=true,
  linkcolor={blue},
  filecolor={Maroon},
  citecolor={Blue},
  urlcolor={Blue},
  pdfcreator={LaTeX via pandoc}}

\title{O mendigo, o moleque e o malandro de João Antônio estão de volta.
Saindo do submundo para a galeria dos heróis marginais}
\author{Leo Gilson Ribeiro}
\date{}

\begin{document}
\maketitle
\begin{abstract}
Jornal da Tarde, 1975-10-4. Aguardando revisão.
\end{abstract}

João Antônio é que atinge lascas mais próximas de medula. Sua classe não
tem classificação social, muito menos perspectivas de ascensão ou
transformação em potenciais integrantes da sociedade que, rotulada ``de
consumo'', na realidade consome os que a consomem: a estrutura
deglutindo a carne e a alma em troca de trinta dinheiros escassos
açulados pela publicidade criadora de necessidades supérfluas.

Já em \emph{Leão de Chácara} o jovem autor paulista se afirmara como um
dos talentos maios violentos, como o diamante bruto que faria a fortuna
de uma literatura agora artificialmente anêmica e desvalida. Com a
reedição oportuna de \emph{Malagueta, Perus e Bacanaço} (Editora
Civilização Brasileira, 159 páginas, coedição barateada pelo Instituto
Nacional do Livro) ele cimenta mais ainda a solidez de sua vocação, a
exuberância fantástica de sua seiva narrativa.

Que não fiquem dúvidas nem ilusões: não se pretende aqui confrontar
autores nem ressaltar superioridades. João Antônio \emph{complementa}
tufo por tufo a mesma zona vegetal dos personagens de Dalton Trevisan.
Só a faixa de terra é que é diferente, de diferentes conotações, de
intenções e naturezas intrinsecamente diversas. Porque o mundo de João
Antônio é o sub, o anti, o infra, o anterior a qualquer \emph{status}
estabelecido, os marginais irmãos dos \emph{beatnicks} e \emph{hippies},
de Céline e Genet, e Kerouac e Ginsberg, de Buñuel, Rossellini e
Fellini.

Portanto, se Dalton Trevisan e na literatura moderna do Brasil o
primeiro \emph{voyant}, como o Baudelaire de uma Curitiba de
\emph{Poèmes en Prose} matizadíssima, João Antônio é seu Rimbaud, João
Antônio convive com as prostitutas, as cafetinas, os invertidos, os
tiras corruptos que se confundem com os ladrões, os malandros, os
jogadores de sinuca e os favelados e mendigos. Todos formigando debaixo
dos viadutos de São Paulo, órfãos em suas noites de luz néon, pedintes
na Lapa, sortudos no bilhar, emersos por horas das cafuas, mazelas da
cidade que os escorraça, com o nariz tapado, para o Juqueri ou a Casa de
Detenção.

Seu longo conto que dá o título ao livro é uma benção. Cai como uma
chuva numa terra já ressequida já há anos de qualquer verdade literária
ou humana.

O percurso dos três malandros, o velho mendigo doente, Malagueta, o
adolescente Perus, que fugiu da cidade poluída pelo cimento da fábrica
de J. J. Abdalia, e Bacanaço, o malandro mulato de anel no dedo que
sonha com uma ``mina'', uma mulher de alto coturno, com ``lordeza'' na
praça, apartamentos, prostituição de luxo, quatro mil cruzeiros por dia
de lucro em sua mão de rufião ladino -- é um percurso da própria
literatura brasileira. Mais: João Antônio já é um porta-voz daquele que
os economistas do Fundo Monetário Internacional classificaram de Quarto
Mundo: as nações da África Negra crescentemente assoladas pela fome, do
deserto do Saara, que avança 5 kms por ano adentro de sua possibilidade
de sobrevivência; do Haiti, da Bolívia, das camadas do Brasil intocadas
pelo milagre econômico. Acompanhar os três é estampar sobre uma tela
cinza a marcha das esperanças alvoroçadas para o ponto inicial e final
-- o ponto zero, o fracasso --, quando se fecha o círculo, geográfico e
vivencialmente, na Lapa. É ouvir uma melodia baça, sutilíssima, que
acompanha os filmes italianos da resignação perante a inelutabilidade da
miséria, um De Sica irmão dos deserdados, mas que só pode expor, sem
mudar, a cama noturna forrada de jornais debaixo de pontes abrigadoras
da chuva.

Há quase necessidade de um glossário: o autor capta essa realidade viva
com seu linguajar -- quizilentos, panca, uma crepe sofrida, caçapa, dar
estia, eu me espianto, uma mulher escanzelada, tropicavam -- e com seus
nomes Praça, Paraná, Detefon, Estilingue, Lincoln, Mãozinha, Carne
Frita. Ele não aprende de fora: ele \emph{convive} com essa realidade,
pulula com ela na rede geral que os colhe a todos.

Porque João Antônio nos força a virar o mundo do avesso. Do outro lado
do bordado cintilante do progresso, das ruas ajardinadas, das famílias
felizes e de casas e salários fixos, há a parte enviesada, com pedaços
de linha aparecendo, fiapos cortados, descontinuidade de cor, fragmentos
súbitos, sem arremate nem beleza estereotipada. João Antônio prossegue a
revolução ético-estética do cinema neorrealista italiano e do
Baudelaire, primeiro a enfocar as sobras humanas da grande cidade. E com
isso ele cria uma poesia e um ritmo novo, inéditos totalmente no Brasil,
apesar de muitos autores terem remexido esse entulho como Aluísio de
Azevedo em \emph{O Cortiço}, Lima Barreto e Antônio de Alcântara
Machado, entre outros.

A diferença está entre uma realidade observada e uma realidade
compartilhada.

É uma prosa poética que continua mais Gregório de Maros Guerra na sua
desmistificação de hipocrisia geral do que o romance do Nordeste, de
meridiana denúncia social. João Antônio impregna suas páginas de um
lirismo contido, mas eloquentíssimo em sua surdina:

``Mas o misticismo da luz elétrica, de um mistério como o deles, só
cobria solidões constantes, vergonhas, carga represada de humilhação,
homens pálidos se arrastando, pouco interessava se eram sapatos de
quatro contos, cada um com seus problemas e sem sua solução e com chope,
bate-papo, xícara retindo café, iam todos juntos, mas ilhados,
recolhidos, como martelo sem cabo. Nem era à toa que aquela dona,
criaturinha magra, mina bem nova ainda, se apagou no bamborete do canto
e trazia nos olhos uma tristeza de cadela mansa\ldots{} Quando a justa,
perua preto-e-branca dos homens da polícia roncava no asfalto, a verdade
geral se punha na maioria dos olhos: Lugar de vagabundo é na Casa de
Detenção''.

João Antônio não tem o ceticismo de um Juan Carlos Onetti nem a
sofisticação de um Manuel Puig, no entanto, é a mais virial e renovadora
presença do conto latino-americano em que se insere, transcendendo os
limites de uma única literatura. Sua visão alia o leitor sem
artificialismos, cheira a Jack London, a Gorki, do Zola de
\emph{Germinal} abrange a urina e a pulga, o punhal e o taco de bilhar,
os japoneses da Liberdade e as lésbicas masculinizadas do crime e da
calçada. Ela traz uma lucidez ao acontecimento múltiplo que é a
literatura social sem trair nenhum elemento desse binômio. Ao contrário:
transcende-os, englobando-se numa compaixão no sentido mais radical e
etimológico do termo, e com fulgurações de uma filosofia religiosa,
cristã, que banhasse os seres humanos de uma luz palpitante. São os
mortos que ele ilumina, os Lázaros e ``lazzaroni'' que ele coloca diante
do espectador.

E cumpre a função mais vital e mais perene da literatura que não prega,
não estetiza, não utiliza: \emph{transforma} o observador, incute-lhe
valores novos, explosivos, não panfletários. Por isso seus livros não
acabam numa visão simplista -- nem ideológica nem estética -- porque
seus livros são estão aí para servir de teses. Eles \emph{são}. E na
plenitude de serem, pela sua própria existência, eles forçosamente
modificam a rotina, ramificam-se do estômago e do coração até o cérebro,
tic-tac finalmente uníssono com a entrada e a saída de ar e de sangue no
pulmão. Sem apaziguar, eles harmonizam disparidades e restituem ao
leitor o ser humano integral, sem ardis, sem artifícios, sem catequeses,
sem apologias morais.

Como toda modificação de ângulo e enquadramento, para uns ele será um
autor incômodo. Para os que pensam e creem como nós numa modificação do
homem sem a brutalidade maceradora dos totalitarismos, João Antônio só
merece a nossa gratidão, nossa admiração e nossa confiante esperança.



\end{document}
